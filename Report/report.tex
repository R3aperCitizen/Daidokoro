\documentclass[a4paper,12pt]{report}
\usepackage{alltt, fancyvrb, url}
\usepackage{graphicx}
\usepackage[utf8]{inputenc}
\usepackage{float}
\usepackage{hyperref}
\usepackage[italian]{babel}
\usepackage[italian]{cleveref}
\title{Relazione su "Daidokoro" \\ Elaborato per il corso di Basi di Dati}

\author
{
    Matteo Giorgini - 0001136576 \\
    Tommaso De Tommaso - 0001077338 \\
    Edoardo Scorza - 0001077424 \\
}
\date{\today}
\begin{document}
\maketitle
\tableofcontents

\chapter{Analisi dei requisiti}
Si vuole realizzare un database a supporto di un social network di ricette.
Questo dovrà immagazzinare dati relativi agli utenti registrati, alle ricette pubblicate, gli ingredienti disponibili, i valori nutrizionali, la categoria nutrizionale, i preferiti, le valutazioni, le collezioni e gli obiettivi.

\section{Intervista}
Si vuole tenere traccia degli \textbf{utenti} registrati salvandone il loro username, la foto profilo, l'email, la password, l'esperienza ed il livello, che vengono usati per premiare un maggiore utilizzo del social network, inoltre viene generato un codice univoco per identificarlo. Un utente, una volta registrato, potrà effettuare l'accesso ed utilizzare il social network osservando le ricette altrui, aggiungendole ai preferiti, salvandole in collezioni o in diete e valutandole con un voto ed un commento. Inoltre l'utente potrà pubblicare le proprie ricette e sbloccare obiettivi.
Una \textbf{ricetta} è caratterizzata da un nome, una foto, una descrizione, una difficoltà e un tempo di realizzazione indicativo, oltre agli ingredienti che la compongono ed i passi necessari a prepararla.
Un \textbf{ingrediente} è caratterizzato da un nome, una descrizione, dai suoi valori nutrizionali e dalla categoria nutrizionale a cui appartiene.
Una \textbf{collezione} è caratterizzata da un nome, una descrizione, una data e le ricette che ne fanno parte. Lo stesso vale per una \textbf{dieta}, tuttavia questa deve anche aderire ad una categoria nutrizionale specifica, vincolando le ricette contenute. Ad esempio, la dieta per i celiaci aderirà alla categoria nutrizionale della celiachia e potrà comprendere solo ricette con ingredienti compatibili.
Un \textbf{obiettivo} è caratterizzato da un nome, una descrizione e dalla quantità di esperienza da dare all'utente una volta sbloccato.

[ SPIEGAZIONE SU COME SBLOCCARE E COSA LIMITANO GLI OBIETTIVI ]

[ ANALISI DELLE VALUTAZIONI ]

\section{Ambiguità}
Secondo sottocapitolo
\section{Definizioni}
Secondo sottocapitolo


\chapter{Progettazione concettuale}
\section{Schema scheletro}
Secondo sottocapitolo
\section{Raffinamenti proposti}
Secondo sottocapitolo
\section{Schema Concettuale}
Secondo sottocapitolo


\chapter{Progettazione logica}
capitolo logica
\section{Volume dei dati}
Secondo sottocapitolo
\section{Operazioni principali e frequenza}
Secondo sottocapitolo
\section{Tabelle degli accessi}
Secondo sottocapitolo
\section{Raffinamento schema}
Secondo sottocapitolo
\section{Analisi rindondanze}
Secondo sottocapitolo
\section{Traduzione di entità e relazioni in associazioni}
Secondo sottocapitolo
\section{Schema relazionale finale}
Secondo sottocapitolo
\section{Traduzione delle operazioni in query SQL}
Secondo sottocapitolo


\chapter{Progettazione applicazione}
\section{Descrizione dell'architettura\\ dell'applicazione realizzata}
Abbiamo sviluppato un'applicazione Social Media di ricette culinarie per dispositivi mobile in linguaggio C\# utilizzando il framework .NET MAUI (utilizzato per la creazione di applicazioni mobile).
Per le connesioni al database viene utilizzata la libreria C\# MySQLConnector e il database risiede in un server MySQL.
Per ogni tabella contenuta nel DB viene creata una corrispettiva classe che la rappresenta all'interno del
model dell'applicazione. Inoltre abbiamo creato una classe DBService che gestisce le connessioni effettuate al database e presenta vari metodi:
\begin{itemize}
    \item TryConnection(DBCredentials dbc): prende in input le credenziali per accedere al database e restituisce un booleano sul risultato della connessione al database.
    \item ExistInTable(string query): restituisce un booleano in base al numero di elementi che produce la query data (0 elementi prodotti = false)
    \item GetData(string query): restituisce una lista di elementi generici in base alla tabella su cui viene effettuata la query e al tipo specificato.
    \item InsertElement(List〈Tuple〈string, object〉〉 values, string query): inserisce gli elementi nella tabella specificata nella query inserendo nei valori della query non settati (?) il rispettivo valore contenuto nella lista di tuple.
    \item RemoveOrUpdateElement(string query): rimuove o aggiorna la riga di una tabella seguendo la query data.
\end{itemize}

\end{document}
