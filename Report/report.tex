\documentclass[a4paper,12pt]{report}
\usepackage{alltt, fancyvrb, url}
\usepackage{graphicx}
\usepackage[utf8]{inputenc}
\usepackage{float}
\usepackage{hyperref}
\usepackage[italian]{babel}
\usepackage[italian]{cleveref}
\title{Relazione su "Daidokoro" \\ Elaborato per il corso di Basi di Dati}

\author
{
    Matteo Giorgini - 0001136576 \\
    Tommaso De Tommaso - 0001077338 \\
    Edoardo Scorza - 0001077424 \\
}
\date{\today}
\begin{document}
\maketitle
\tableofcontents

\chapter{Analisi dei requisiti}
Si vuole realizzare un database a supporto di un social network di ricette.
Questo dovrà immagazzinare dati relativi agli utenti registrati, alle ricette pubblicate, gli ingredienti disponibili, i valori nutrizionali, la categoria nutrizionale, i preferiti, le valutazioni, le collezioni e gli obiettivi.

\section{Intervista}
Si vuole tenere traccia degli utenti registrati salvandone il loro username, email, password, esperienza, livello e foto profilo, inoltre viene generato un codice per identificarlo.

\section{Ambiguità}
Secondo sottocapitolo
\section{Definizioni}
Secondo sottocapitolo


\chapter{Progettazione concettuale}
\section{Schema scheletro}
Secondo sottocapitolo
\section{Raffinamenti proposti}
Secondo sottocapitolo
\section{Schema Concettuale}
Secondo sottocapitolo


\chapter{Progettazione logica}
capitolo logica
\section{Volume dei dati}
Secondo sottocapitolo
\section{Operazioni Principali e Frequenza}
Secondo sottocapitolo
\section{Tabelle degli accessi}
Secondo sottocapitolo
\section{Raffinamento Schema}
Secondo sottocapitolo
\section{Analisi Rindondanze}
Secondo sottocapitolo
\section{Traduzione di entità e relazioni in associazioni}
Secondo sottocapitolo
\section{Schema relazionale finale}
Secondo sottocapitolo
\section{Traduzione delle operazioni in query sql}
Secondo sottocapitolo


\chapter{Progettazione applicazione}
capitolo logica
\section{Piattaforma di sviluppo}
sezione
\section{Architettura}
sezione
\end{document}
