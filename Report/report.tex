\documentclass[a4paper,12pt]{report}
\usepackage{alltt, fancyvrb, url}
\usepackage{graphicx}
\usepackage[utf8]{inputenc}
\usepackage{float}
\usepackage{hyperref}
\usepackage[italian]{babel}
\usepackage[italian]{cleveref}
\title{Relazione su "Daidokoro" \\ Elaborato per il corso di Basi di Dati}

\author
{
    Matteo Giorgini - 0001136576 \\
    Tommaso De Tommaso - 0001077338 \\
    Edoardo Scorza - 0001077424 \\
}
\date{\today}
\begin{document}
\maketitle
\tableofcontents

\chapter{Analisi dei requisiti}
Si vuole realizzare un database a supporto di un social network di ricette.
Questo dovrà immagazzinare dati relativi agli utenti registrati, alle ricette pubblicate, gli ingredienti disponibili, i valori nutrizionali, la categoria nutrizionale, i preferiti, le valutazioni, le collezioni e gli obiettivi.

\section{Intervista}
Si vuole tenere traccia degli \textbf{utenti} registrati salvandone il loro username, la foto profilo, l'email, la password, l'esperienza ed il livello, che vengono usati per premiare un maggiore utilizzo del social network, inoltre viene generato un codice univoco per identificarlo. Un utente, una volta registrato, potrà effettuare l'accesso ed utilizzare il social network osservando le ricette altrui, aggiungendole ai preferiti, salvandole in collezioni o in diete e valutandole con un voto ed un commento. Inoltre l'utente potrà pubblicare le proprie ricette e sbloccare obiettivi.
Una \textbf{ricetta} è caratterizzata da un nome, una foto, una descrizione, una difficoltà e un tempo di realizzazione indicativo, oltre agli ingredienti che la compongono ed i passi necessari a prepararla.
Un \textbf{ingrediente} è caratterizzato da un nome, una descrizione, dai suoi valori nutrizionali e dalla categoria nutrizionale a cui appartiene.
Una \textbf{collezione} è caratterizzata da un nome, una descrizione, una data e le ricette che ne fanno parte. Lo stesso vale per una \textbf{dieta}, tuttavia questa deve anche aderire ad una categoria nutrizionale specifica, vincolando le ricette contenute. Ad esempio, la dieta per i celiaci aderirà alla categoria nutrizionale della celiachia e potrà comprendere solo ricette con ingredienti compatibili.
Un \textbf{obiettivo} è caratterizzato da un nome, una descrizione e dalla quantità di esperienza da dare all'utente una volta sbloccato.

[ SPIEGAZIONE SU COME SBLOCCARE E COSA LIMITANO GLI OBIETTIVI ]

[ ANALISI DELLE VALUTAZIONI ]

\section{Ambiguità}
Secondo sottocapitolo
\section{Definizioni}
Secondo sottocapitolo


\chapter{Progettazione concettuale}
\section{Schema scheletro}
Secondo sottocapitolo
\section{Raffinamenti proposti}
Secondo sottocapitolo
\section{Schema Concettuale}
Secondo sottocapitolo


\chapter{Progettazione logica}
capitolo logica
\section{Volume dei dati}
Secondo sottocapitolo
\section{Operazioni principali e frequenza}
Secondo sottocapitolo
\section{Tabelle degli accessi}
Secondo sottocapitolo
\section{Raffinamento schema}
Secondo sottocapitolo
\section{Analisi rindondanze}
Secondo sottocapitolo
\section{Traduzione di entità e relazioni in associazioni}
Secondo sottocapitolo
\section{Schema relazionale finale}
Secondo sottocapitolo
\section{Traduzione delle operazioni in query SQL}
Secondo sottocapitolo


\chapter{Progettazione applicazione}
capitolo logica
\section{Piattaforma di sviluppo}
sezione
\section{Architettura}
sezione
\end{document}
